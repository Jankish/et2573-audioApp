\section{Part 1}
\emph{Voice Activity Detector} (VAD) is a technique used in signal
processing to detect the presence of human voice in a signal. It can be
an energy detector that indicates speech when the energy of the filtered signal 
exceeds a predefined threshold. Considered an important technology in speech 
based communication, today there are various types of applications that use it. 
Therefore a wide variety of VAD algorithms have been developed to provide the needed features. 

There are different kind of stand-alone commercial baby monitors on the market today. 
From the most basic, that use one-way radio communication, to advance two-way communication
monitors that use signal processing to transmit audio when a predefined threshold has been 
reached. It is also possible to find baby video monitors that broadcast both audio and video when 
the sensors notice movement. Since most of the monitor applications rely on radio signals to 
communicate between the units there is a probability that the signal 
will weaken or possibly not even reach the receiver because it needs to pass through multiple walls
of varying thickness. The signal could also be effected by other applications. 
As the stand-alone monitor focuses on reliability (among other important sale strategies), 
little is known about the security features. It is possible to assume that the communication 
is uncrypted, at least in some products, and therefore introduces a potential risk for 
intrusion of peoples privacy. 

To resolve the issues brought up above, an application such as the \emph{Baby Activity Detector} (BAD) can can be made more 
portable, versatile and secure with the help of todays smart-phone technology and VAD. There are
many VAD algorithms to choose from and they all have their strengths and weaknesses. Complex 
algorithms such as Linear Predictive coding (LPC), mel-frequency cepstrum (MFC) are very powerful
but quite difficult to grasp and also to implement, they can be considered out of scope for 
this course. The following VAD algorithms are easy to implement and can be, when combined, quite robust 
for the task of a basic BAD. The simple short-time energy algorithm calculates the energy levels 
for each frame to detect voice, unvoiced or silenced regions. Voiced regions will have higher 
energy levels, however, the algorithm does not take unwanted noise into account which means 
that we can have false indication of voice detection. In order to remove the noise from the 
signal, spectral subtraction can be preformed. In the case of BAD, the threshold needs to be 
adjusted so that unforeseeable sound is not interpret as the infants cry. 
Zero-crossing rate (ZCR), is the rate at which a signal changes from plus to minus and back. 
The higher the rate the higher the frequency which indicates possible voice activity. According to \cite{infantDetec} 
the cry sound that an infant makes has a fundamental frequency of 250-600 Hz (pitch). To be able to
use ZCR together with the information above, it is necessary to extract the pitch from the signal
in order to match the frequency interval. 

The main task of BAD is to detect infant activity, an alternate algorithm is proposed in \cite{infantDetec}. 
It describes an cry detection algorithm that is build up by three main stages. i) \emph{VAD}, a 
statistical model-based detector \cite{statis} is used for detecting sections with sufficient audio acitivity. It also helps
to reduce the power consumption. ii) \emph{Classification}, uses k-nearest neighbours (k-NN) algorithm \cite{k-NN} 
to label each frame as either 'cry' (1), close enough, or 'no cry' (0). iii) Post-processing, 
validation stage in order to reduce false-negative errors. The idea of having devoted algorithm to detect
infant cry is a winning concept for a BAD application, according to the authors it even had promising results 
in low SNR. Despite simplicity of the algorithm many of the features required to implement were mentioned
earlier to be out of scope for this project.

An algorithm that might be of interest \cite{blekinge} suggests a new approach to speech enhancement, 
without the help of VAD technology. The signal is divided into multiple sub-bands and an noise floor 
level estimate is calculated simultaneously as the short-time average. The goal is to boost the 
sub-bands with high Signal-to-Noise Ration (SNR) instead of to suppress the lower. This algorithm 
has great potential to reduce the noise levels when analyzing incoming signals to the BAD application. 

A quick search on the net gives significant amount of hits for smart-phone based BAD applications. The techniques 
vary, from bluetooth to Wi-Fi and 3-/4G solutions. The award winning application \emph{Baby Monitor 3G} 
\cite{bm3G} is a feature rich cross platform application that solves most the of issues brought up in this report. It supports 
both Wi-Fi and 3G/LTE networks, ability to transfer high quality live video, adjustable microphone sensitivity, 
talk-back functionality and guarantees both reliability and privacy. It can be assumed that the Android based BAD 
application \emph{Dormi} \cite{sleekbit} offers similar features as Baby Monitor 3G, even if Baby Monitor 3G's feature list 
provides barely any deeper information. It is noteworthy that \emph{Domri} have both \emph{Smart noise detection} 
and \emph{Adaptive audio enhancement} as sales pitch, which from a engineering point of view is very attractive. 

Following is a purposed algorithm for an BAD application
\newpage
\begin{verbatim}
while(true){
    % Record audio can place it into the register %
    Get frame from the register 
    Divide the signal into different sub-bands with FFT
    Calculate the total short-time energy average  
    Calculate noise level for each sub-bands
    if energy average above threshold
        Calculate the gain for each sub-band
        Boost the sub-band with high SNR
        % Extract fundamental frequency (pitch)
        Count the ZCR under 1 seconds
            if the ZCR is within 250-600 Hz 
                Possible infant activity detected!
                (Occurs only first time, and needs to be reseted)
		% Send frames %
                Start broadcasting the sound to the receiver    
            end
    else
        Reset ZCR
        Dismiss and get next frame from register
    end
}
\end{verbatim}
