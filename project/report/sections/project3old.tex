\newpage
\section{Part 3 (4)}
The last section of this report, \emph{Part 3}, actually represents \emph{Part 4}
in the "step-by-step" instructions provided from the institute.

As mentioned in \emph{Part 2}, the \emph{recursive averaging} algorithm is
crucial for both the simple and advanced algorithm. An comprehensive study of 
the framework was preformed before the implementation of the simple algorithm
started.

\subsection{StartFragment.java}
The first modifications were performed in \emph{startFragment}'s,
\emph{run()} method. 

\lstinputlisting[language=Java]{/Users/daniel/Documents/et2573-audioApp/project/report/sections/StartFragment.java}

To resemble the MATLAB implementation, the buffer length was decreased from 160
to 10. Dividing the \emph{sum} with \emph{buffer.length} was kept to scale down 
the new \emph{P(n)}. After calculating the recursive averaging, the 
\emph{recursiveSum} is sent to \emph{updateStates()} as parameter.

\subsection{StudentDetector.java}
The Java class \emph{TestDetector}, which implemented the 
\emph{BabyDetector} interface, was renamed to \emph{StudentDetector}. Since the 
class is more complex than the one above, more thorough explanation is provided 
for each method.

\lstinputlisting[language=Java]{/Users/daniel/Documents/et2573-audioApp/project/report/sections/StudentDetector0.java}

The most important, in the code above, are the three constants
\emph{maxFrames}, \emph{framesThreshold}, and \emph{MAX ENERGY CEILING}.

\begin{itemize}
\item \emph{maxFrames}
	
	Defines how many frames should be used to average the noise level 
	prior to start of the main application.
\item \emph{framesThreshold}
	
	Defines how many frames need to breach the threshold value before
	setting off the alarm.
\item \emph{MAX ENERGY CEILING} 
	
	Despite the questionable variable name, this defines the highest 
	possible threshold value.
\end{itemize}
The \emph{MAX ENERGY CEILING} and \emph{framesThreshold} constant were selected 
after evaluating the energy levels in different test environments. 
\emph{framesThreshold} was given an higher value, compared to the MATLAB
implementation, because it lowered the probability of false alarms.
Since the application is interested in the background noise levels when the
caretaker is not present in the room.  Fairly long initialization time was
implemented, hence why \emph{maxFrames} was set to a high value. 

\lstinputlisting[language=Java]{/Users/daniel/Documents/et2573-audioApp/project/report/sections/StudentDetector2.java}

\emph{updateState()} method will be called from \emph{StartFramgment} class and
since the user has selected the student detector, the \emph{StudentDetector} 
inherited method will be called. The input parameter for this method is the 
average sum, \emph{recursiveSum}. The code is divided into two parts, the upper
if-statement, that takes the boolean value \emph{init}. 
When \emph{init == false}, the background noise levels is averaged and a 
threshold value is configured. This is considered to be performed when starting 
the application, which is why \emph{init = true} when initialization is finished.

The lower part, when \emph{init == true}, is the main activity detector that
validates if the \emph{currentValue} is bigger than threshold. However, if the
sensitiivty bar has been adjusted (\emph{senseChange == true}), a new scaled 
threshold value will be returned from \emph{getThreshold()}.


\lstinputlisting[language=Java]{/Users/daniel/Documents/et2573-audioApp/project/report/sections/StudentDetector3.java}

The \emph{getThreshold()} method calculates the threshold value given the
baseline and percentage value. Initially, the sensitivity bar will be 0 and
the $case$ 0 will be \emph{MAX ENERGY CEILING} since \emph{multiply} is
extracted from
\[
\frac{\mbox{\emph{MAX ENERGY CEILING}}}{baseline} 
\]
The $case$ 100 is suppose to return the baseline level. However, this is 
considered too sensitive. Instead a small fraction is added to baseline 
before it is returned. In the $default$ $case$ a proper estimation is 
calculated in order to match the scaling set by the user.


\lstinputlisting[language=Java]{/Users/daniel/Documents/et2573-audioApp/project/report/sections/StudentDetector5.java}

The inherited method \emph{getConfigurationView} is inspired by the two provided
classes that also implement \emph{BabyDetector} interface. When the user adjusts
the sensitivity bar, the flag \emph{senseChange} will be set to \emph{true} and
threshold will receive a new value.

\subsection{Advanced algorithm}
After completing the implementation of the simple algorithm, attempts were made
on implementing the advanced algorithm. However, time being a factor, the poorly
documented third party DSP.jar packaged required more time and more adjustments
in the framework to be able to work. The decision was made not to proceed with
implementing the advance algorithm.
