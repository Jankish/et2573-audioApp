\newpage
\section{Part 3 (4)}
The last section of this report, \emph{Part 3}, actually represents \emph{Part 4}
in the "step-by-step" instructions provided from the institute.

As mentioned in \emph{Part 2}, the \emph{recursive averaging} algorithm was
crucial for both the simple and advanced algorithm. The implementation required
comprehensive study of the framework that was provided since the author of this
report had no previous experience of Android application developement. 

The first modification was performed in \emph{startFragment.java},
\emph{run()} method. 

\lstinputlisting[language=Java]{/Users/daniel/Documents/et2573-audioApp/project/report/sections/StartFragment.java}

To resemble the MATLAB implementation the buffer length was decreased from 160
to 10.  While MATLAB used \emph{floating point} precision when calculating the 
algorithm, the Android application was written with \emph{short} precision, 
assumingly to avoid overloading the CPU.
