\newpage
\section{Part 3 (4)}
The last section of this report, represents \emph{Part 4} in the "step-by-step"
instructions provided by the institute.As mentioned in \emph{Part 2}, the 
\emph{recursive averaging} is crucial for both the simple and advanced algorithm. 
For a more rigid and maintainable application, a comprehensive study of the 
framework was performed prior to theimplementation. 

In order to make the student \emph{Baby Activity Detector} work, additional
methods were added to existing interface and classes. The added methods will
first be presented and the implementation will follow.

\subsection{BabyDetector.java}
Three new methods were added to the \emph{BabyDetector} interface in order to
make it possible to change the boolean variable \emph{init} and reset the 
\emph{frameCounter \& sum} in \emph{StudentDetector} from 
\emph{StartFragment}.  

\lstinputlisting[language=Java,
	linerange=61-77]{/Users/daniel/Documents/et2573-audioApp/project/report/sections/BabyDetector.java}

\subsection{BabyState.java}
To notify the user when the application is initializing or has
triggered off the alarm, a fourth state was introduced to the \emph{BabyState}
enum class. 

\lstinputlisting[language=Java,linerange=3-10]{/Users/daniel/Documents/et2573-audioApp/project/report/sections/BabyState.java}

\subsection{StartFragment.java}
As the name suggests, a lot is start from this class. The private \emph{Audio}
class which extends \emph{Thread}, is is responsible for audio handling. The
simple algorithm is partially implemented in the private class, only
the \emph{recursive algorithm}. The rest of the algorithm is in
\emph{StudentDetector}. 

\lstinputlisting[language=Java,linerange={32-36}]{/Users/daniel/Documents/et2573-audioApp/project/report/sections/StartFragment.java}

To resemble the MATLAB implementation, the buffer length was decreased from 160
to 10. Dividing the \emph{sum} with \emph{buffer.length} was kept to scale down 
the new \emph{P(n)}. After calculating the \emph{recursive averaging}, the 
\emph{recursiveSum} is sent to \emph{StudentDetector}'s \emph{updateStates()} 
method as parameter.
\lstinputlisting[language=Java,linerange={132-155}]{/Users/daniel/Documents/et2573-audioApp/project/report/sections/StartFragment.java}

The new state, that was implemented in \emph{BabyState}, is used to decide
what to display to the user. 

\lstinputlisting[language=Java,linerange={173-190}]{/Users/daniel/Documents/et2573-audioApp/project/report/sections/StartFragment.java}

If the application is stopped at any time, after it has started, it will need to
recalculate the baseline (noise level). That means that \emph{init} will need to
be set to \emph{false} and \emph{frameCounter \& sum} need to be reset. That is
done by calling \emph{reset()}. However, if the application is stopped before
the initialization is completed the boolean variable, \emph{init}, needs to be 
set to \emph{false} through \emph{setInit(false)} call. 

\lstinputlisting[language=Java,linerange={61-91}]{/Users/daniel/Documents/et2573-audioApp/project/report/sections/StartFragment.java}


\subsection{StudentDetector.java}
The Java class \emph{TestDetector}, which implemented the \emph{BabyDetector}
interface, was renamed to \emph{StudentDetector}. Since the class is unfamiliar
to the reader, a more thorough explanation is provided for important methods and
variable names.

\subsection{Private variables}
Many of the private variables declared in \emph{StudentDetector} are self
explanatory. However, some of them are important  need further clarifying.
\lstinputlisting[language=Java,linerange={9-10,12-14,16-21,23-27,29-31}]{/Users/daniel/Documents/et2573-audioApp/project/report/sections/StudentDetector.java}
\begin{itemize}
\item \emph{init}
	
	Boolean value that keeps track if the initializing sequence has been
	performed.
\item \emph{senseChange}

	Boolean value that keeps track if the user has adjusted the sensitivity
	bar in the configuration menu.
\item \emph{maxFrames}

	Defines how many frames should be used to average the baseline (noise
	level) prior to the start of the student detector application.
\item \emph{framesThreshold}

	Defines how many consecutive frames need to breach the threshold value
	before setting off the alarm.
\item \emph{MAX ENERGY CEILING}
	
	Despite the questionable variable name, this defines the highest
	possible threshold value.
\end{itemize}

The \emph{MAX ENERGY CEILING} and \emph{framesThreshold} constants were selected
after evaluating the energy levels in different environments.
\emph{framesThreshold} was given an higher value, compared to the MATLAB
implementation, because it lowered the probability of false alarms. Since the
application is interested in the background noise levels, a fairly long
initialization time was set to \emph{maxFrames}. Another benefit is that it
gives the caretaker time to start the application and leave the room without
setting off the alarm.

\subsection{updateState()}
\emph{updateState()} method is called from \emph{StartFragment} class since it
inheritage the \emph{BabyDetector} interface. The input parameter for this
method is the recursive averaging sum, \emph{recursiveSum}. It is possible to
visualize the method as two parts. The upper, when the boolean value \emph{init}
is false and the lower part, when \emph{init} is true. When \emph{init} is
false, the background noise levels are averaged and a threshold value is
configured. This is considered to be performed before the detector is activated,
which is why the \emph{init} is set to true when initialization is finished. If
the initialization is aborted or the application is restarted, the \emph{init}
value should be set to false and a new baseline should be calculated.
\lstinputlisting[language=Java,linerange={32-40,42-44,46-71}]{/Users/daniel/Documents/et2573-audioApp/project/report/sections/StudentDetector.java}
The lower part of the method, when \emph{init} equals true, is the main activity
detector that validates if the \emph{currentValue} is bigger than the threshold.
However, if the sensitivity bar has been adjusted (\emph{senseChange}) is set to
true), a new scaled threshold value will be returned from \emph{getThreshold()}.

\subsection{getThreshold()}
The \emph{getThreshold()} method calculates the threshold value given the
baseline and percentage. Initially, the sensitivity bar will be set to 0 \% and 
the $case$ 0 will be $\sim$\emph{MAX ENERGY CEILING} since \emph{multiply} is 
extracted from 
\[
\frac{\mbox{\emph{MAX ENERGY CEILING}}}{baseline}
\]
The $case$ 100 is suppose to return the baseline. However, this is considered
too sensitive. Instead a small fraction is added to baseline before it is
returned. In the $default case$ a proper percentage estimation is calculated in 
order to match the scaling set by the user.
\lstinputlisting[language=Java,linerange={80-97}]{/Users/daniel/Documents/et2573-audioApp/project/report/sections/StudentDetector.java}

\subsection{Advanced algorithm}
After completing the implementation of the simple algorithm, attempts were made
on implementing the advanced. However, time being a factor, together with poor
documentation on the third-party DSP.jar packaged required more time and more
adjustments in the framework to make the bandpass filter work. The decision was
made not to proceed with implementing the advanced algorithm since the simple
was good enough for a student release.
